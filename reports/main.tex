\documentclass{letter}
\usepackage{chngpage}
\usepackage{enumitem}
\usepackage{geometry}
 \geometry{
 a4paper,
 total={210mm,297mm},
 left=20mm,
 top=20mm,
 right=20mm,
 bottom=20mm
 }
\usepackage{graphicx}
\usepackage{hyperref}
\usepackage[utf8]{inputenc}
\usepackage[scaled]{helvet}
\usepackage{tabularx}

\title{NGS Report 1-0-a}
\author{Joris Argentin}
\date{January 2023}

%Constant
\newcommand\NGSLibrairie{SureSelect QXT FH\_Angers\_v1.0}
\newcommand\NGSSequenceur{Illumina NextSeq500 2x150bp}

%Mutations are extracted from the mutations .csv file

%Constant
%Could be extracted from a .json file, as Pierre G. does
\newcommand\BioinfoFoxNGSVersion{1.0.0}
\newcommand\BioinfoGenome{GRCh37 (hg19, 2013)}
\newcommand\BioinfoDemultiplexingTool{blc2fastq}
\newcommand\BioinfoDemultiplexingVersion{2.20.0.422}
\newcommand\BioinfoAlignmentTool{bwa-mem}
\newcommand\BioinfoAlignmentVersion{0.7.17-r1188}
\newcommand\BioinfoVariantCallingATool{HaplotypeCaller}
\newcommand\BioinfoVariantCallingBTool{Mutect2}
\newcommand\BioinfoVariantCallingCTool{Varscan}
\newcommand\BioinfoVariantCallingDTool{Pindel}
\newcommand\BioinfoVariantCallingAVersion{4.2.6.1}
\newcommand\BioinfoVariantCallingBVersion{4.2.6.1}
\newcommand\BioinfoVariantCallingCVersion{2.3.9}
\newcommand\BioinfoVariantCallingDVersion{0.2.5b9}
\newcommand\BioinfoClassificationTool{Annovar}
\newcommand\BioinfoClassificationVersion{2020-06-07}
\newcommand\BioinfoAnnotationList{COSMIC (v.89 \& v.92), ClinVar (2020-03-16), GnomAD (Exome v2.1), avsnp138, IARC}
\newcommand\BioinfoSeuilVAF{2}
\newcommand\BioinfoSeuilAF{1}
\newcommand\BioinfoSeuilPolymorphism{0.1}

\newcommand\tab[1][0.5cm]{\hspace*{#1}}

\renewcommand\familydefault{\sfdefault}

\begin{document}
    \begin{minipage}[t][3cm]{0.5\textwidth}
        \vspace{0pt}
        \includegraphics[height=2.5cm]{logo-chu.png}
    \end{minipage}    
    \begin{minipage}[t]{0.5\textwidth}
        \vspace{0pt}
        \begin{tabular}{@{}l@{}}CHU d'Angers\tab Laboratoire d'H\'ematologie - Pr Valérie Ugo \\
        UF Biologie Moléculaire - Dr Odile Blanchet \\
        Dr Damien Luque Paz - Dr Anne Bouvier - Dr Laurane Cottin \\
        4, Rue Larrey 49000 Angers \\
        Tel. secr\'etariat : 0241355353\end{tabular}
    \end{minipage}

\medskip

    \fbox{\parbox{\dimexpr\linewidth-2\fboxsep-2\fboxrule\relax}{\centering \textbf{RECHERCHE DE MUTATIONS DANS LES HEMOPATHIES MYELOIDES PAR SEQUENCAGE DE NOUVELLE GENERATION}}}

    \begin{flushright}
        \Agent \\
        \Prescripteur
    \end{flushright}

    \noindent\begin{tabularx}{\linewidth}{@{}>{\hsize=.5\hsize}X|>{\hsize=.5\hsize}X@{}}

    \textbf{Identit\'e}
    
        \begin{itemize}[label={}]
            \item Nom : \PatientNom
            \item Pr\'enom : \PatientPrenom
            \item Date de naissance : \PatientDDN
            \item Sexe : \PatientSexe
        \end{itemize}   
    &
        \textbf{Informations cliniques}
        
        \begin{itemize}[label={}]
            \item Date de prescription : \Datedeprescription
            \item Date de pr\'el\`evemement : \Datepvt
            \item Type de pr\'el\`evement : \BMHVARIABLEMAT
            \item Taux de Blaste : \Tauxdeblastes\%
            \item Stade de la maladie : \Pathologie
        \end{itemize}
    \end{tabularx}

    \textbf{Indication} :\\
    \IndicationNGS

    \bigskip

    {\large\textbf{Variation} :}
    
    \smallskip
    \begin{center}
        \begin{adjustwidth}{-1cm}{-1cm}
            \begin{tabularx}{\linewidth}{| c | c | X | X | X | c | c | c |}
                \hline
                G\`ene & Transcrit & Variant (c.) & Variant (p.) & Classification & VAF (\%) & Profondeur & Annotation \\
                \hline
            \end{tabularx}
        \end{adjustwidth}
    \end{center}

    \bigskip
    
    \textbf{Conclusion} : \\
    
    \smallskip
    
    \TextField[multiline, bordercolor={}, backgroundcolor={}, width=0.90\linewidth,
    height=5\baselineskip]{ }

    \medskip


    Dr Damien Luque Paz \tab \tab Dr Anne Bouvier \tab \tab \tab \tab Dr Laurane Cottin
    
    \newpage
    
    \textbf{S\'equençage NGS}
    \begin{itemize}[label={}]
        \item Librairie : \NGSLibrairie
        \item Sequenceur : \NGSSequenceur
    \end{itemize}

    \bigskip

    \textbf{Analyse Bioinformatique : FoxNGS-H, v\BioinfoFoxNGSVersion}
    \begin{itemize}[label={}]
        \item Genome de reference : \BioinfoGenome
        \item Demultiplexage : \BioinfoDemultiplexingTool, v\BioinfoDemultiplexingVersion
        \item Alignement : \BioinfoAlignmentTool, v\BioinfoAlignmentVersion
        \item Appel de variants : 
        \begin{itemize}
            \item \BioinfoVariantCallingATool, v.\BioinfoVariantCallingAVersion
            \item \BioinfoVariantCallingBTool, v.\BioinfoVariantCallingBVersion
            \item \BioinfoVariantCallingCTool,  v.\BioinfoVariantCallingCVersion
            \item\BioinfoVariantCallingDTool, v.\BioinfoVariantCallingDVersion
        \end{itemize}
        \item Classification : \BioinfoClassificationTool (\BioinfoClassificationVersion)
        \item Annotation : \BioinfoAnnotationList
        \item Seuil de d\'etection : Fr\'equence all\'elique : \BioinfoSeuilVAF\%
    \end{itemize}
        \begin{center}
            Les variants avec une fr\'equence all\'elique $\geq$ \BioinfoSeuilVAF\% \\
            Les variants consid\'er\'es comme des polymorphismes rares (MAF $\geq$ \BioinfoSeuilPolymorphism\%) ne sont pas report\'es
        \end{center}

    \bigskip
    
    \textbf{Information}

    Panel de g\`enes :
    
    \fbox{\parbox{\dimexpr\linewidth-2\fboxsep-2\fboxrule\relax}{ABL1, ACD, ANKRD26, ASXL1, ASXL2, ATM, ATRX, BCL11A, BCOR, BCORL1, CALR, CBL, CEBPa, CHEK2, CSF3R, CTCF, CUX1, DDX41, DNMT3A, DOT1L, EED, ETNK1, ETV6, EZH2, FBXW7, FLT3, GATA1, GATA2, GNAS, HMGA2, HRAS, IDH1, IDH2, IKZF1, JAK1, JAK2, KDM6A, KIT, KMT2A, KMT2B, KMT2C, KMT2D, KRAS, MAD1L1, MPL, NF1, NFE2, NOTCH1, NPM1, NRAS, PAX5, PDS5B, PHF6, PPM1D, PTEN, PTPN11, RAD21, RUNX1, SETBP1, SF3B1, SH2B3, SMC1A, SMC3, SRP72, SRSF2, STAG1, STAG2, SUZ12, TERC, TERT, TET2, TP53, U2AF1, U2AF2, UBA1, WT1, ZBTB33, ZRSR2}}
    
    Tous les g\`enes du panel ont \'et\'e analys\'es.
    
    Profondeur m\'ediane de s\'equençage : \MedianDepth x \tab Couverture $\ge$200x par position : \CoverageSEQDeuxCentX\%
    
    Compte rendu NGS, version du 16/02/2023

\end{document}
